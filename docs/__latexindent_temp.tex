\documentclass[12pt,a4paper]{article}

	\usepackage[brazil]{babel}
	\usepackage[utf8]{inputenc}
	\usepackage[T1]{fontenc}
	\title{O Sistema \LaTeX}
	
	\begin{document}
	\maketitle
	A ideia central do \LaTeX\ é distanciar o autor
	o máximo possível da apresentação visual da informação.
	
	Ao invés de trabalhar com ideias visuais, o usuário é
	encorajado a trabalhar com conceitos mais lógicos --- e,
	consequentemente, independente da apresentação --- como capítulos,
	seções, ênfase e tabelas, sem contudo impedir o usuário da
	liberdade de indicar, expressamente, declarações de formatação.
	
	A versão mais recente é a \LaTeXe.
	
	% Isto é um comentário que não será processado. Ele serve apenas
	% para fazer anotações não incluídas no resultado final. Atenção
	% ao símbolo do comentário: porcentagem (%).
	A seguir, a fórmula das combinações como um exemplo simplório
	da capacidade matemática do \LaTeX:
	
	\begin{eqnarray}
	C \left (k^n\right) &=& \frac{n!}{k!\cdot(n-k)!}
	\end{eqnarray}
	  
	\end{document}